\documentclass{report}
\usepackage[brazil]{babel}
\usepackage{enumitem}
\usepackage{xcolor}

\title{Resumo Prova 2\\ Segurança da Informação}

\begin{document}
	\maketitle
	
	
	\section{Aulas}
	
	\subsection{Aula 5.0 - Ataques Maliciosos, Ameaças e Vulnerabilidades}
	
	\begin{enumerate}[label=--]
		
		\subsubsection{Segurança está na proteção de ativos de algum invasor}
		\item Ativo é qualquer item que tenha valor para uma organização, sendo eles:
		\item \textbf{TI e infraestrutura de rede} - Hardware, software e serviços.
		\item \textbf{Propriedade intelectual} - Dados confidenciais como patentes, código-fonte, fórmulas ou projetos de engenharia.
		\item \textbf{Finanças e dados financeiros} - Contas bancárias, dados de cartão de crédito e de transações
		financeiras.
		\item \textbf{Disponibilidade e produtividade de serviços} - A capacidade de serviços computacionais e de
		software em dar suporte à produtividade para humanos e máquinas
		\item \textbf{Reputação} - Conformidade corporativa e imagem da marca.
		
		\subsubsection{O termo Hacker pode ser dividido em 3 categorias, sendo elas:}
		\item \textbf{Hacker Black-Hat} : Alguém que invade sistemas de forma maliciosa, para roubar informações e obter algum ganho pessoal. Eles não tem a permissão para acessar os sistemas.		
		\item \textbf{Hacker White-Hat} : Conhecidos também como hackers éticos. São pessoas contratadas para identificar falhas no sistema através de testes de invasão. Depois dos testes é gerado um relatório das vulnerabilidade existes e repassado para a organização.
		\item \textbf{Hacker Gray-Hat} : São os hacker que ficam entre o Black-hat e White-hat. É alguém que invade os sistemas sem permissão, mas geralmente sem intenção maliciosa, apenas para explorar as vulnerabilidades. Podem divulgar o não as vulnerabilidade existes para a organização.
		\item Já um \textbf{Cracker} é alguém com intenção hostil, possui habilidade sofisticadas e pode estar interessado em ganho financeiro.
		\item Existe também os chamados \textbf{Script Kiddie} que são pessoas com pouca habilidade, que apenas seguem seguem instruções para realizar um ataque
		
		\subsubsection{Ferramentas utilizadas por indivíduos maliciosos}
		\item Varredura de vulnerabilidades(Scanner), Programas de varredura de portas, Farejadores(Sniffers), Programa para captura de teclado(Keyloggers).
		
		\subsubsection{O que é uma brecha de segurança}
		\item Qualquer evento que resulte em uma violação de qualquer um dos princípios de segurança é uma brecha de segurança
		\item Pode ser acidental ou maliciosa, e pode afetar a capacidade de uma organização realizar negócios.
		\item \textbf{Common Vulnerabilities and Exposures - CVE} - Sistema de nomenclatura e identificação de vulnerabilidades de segurança em software, mantido pela Mitre Corporation e padroniza a referência de vulnerabilidades. Quando uma vulnerabilidade é descoberta ela recebe um número de identificação único no CVE. Ex: CVE-2021-1234.
		
		\subsubsection{Atividades que podem causar uma brecha de segurança}
		\item Ataques de negação de serviço, onde é apenas um computador realizando o ataque\textbf{(Denial of Service - DOS)}
		\item Ataques de negação de serviço distribuidos onde são vários computadores realizando o ataque a um serviço \textbf{Distributed Denial of Service - DDoS}
		\subitem Ataques de negação de serviço podem ser \textbf{Lógicos}  - Usam falhas de software para arruinar o desempenho de servidores remotos. Ou \textbf{Inundação} - Compromentem a CPU, memória e recursos de rede do computador-vítima com o envio de pacotes SYN.
		\item Comportamento inaceitável de navegador web.
		\item Uso de backdoor(porta de entrada) para acessar recursos.
		\item Modificações acidentais em dados.
		\item \textbf{SPAM} é uma mensagem de e-mail ou mensagens instantâneas indesejadas, basicamente contem anúncios comerciais.
		\item \textbf{Hoax} ou (boato) é um ato com intenção de enganar alguém ou defraudar o receptor.
		\item \textbf{Cookies} é um arquivo com detalhes colhidos em visitas anteriores a um sítio web. Pode conter nome de usuários, informações pessoais e outros.
		
		\subsubsection{Vulnerabilidades e Ameaças}
		\item Uma \textbf{Ameaça} é qualquer ação que possa danificar um ativo.
		\subsubitem Ameaças mais comuns são: Software malicioso, Falha de hardware ou software, atacante interno, roubo de equipamento, atacante externo, desastre natural, espionagem industrial, terrorismo 
		\item Uma \textbf{Vulnerabilidade} é qualquer ponto fraco em um sistema que possibilite que uma ameaça causa danos a ele.
		
		\subsubsection{Ataque}
		\item Existem 4 categorias de ataque
		\item \textbf{Fabricação} - Criação de uma fraude de modo a enganar o usuário
		\item \textbf{Interceptação} - Escuta transmissões e redireciona para uso não autorizado.
		\item \textbf{Interrupção} - Causa uma quebra em um canal de comunicação
		\item \textbf{Modificação} - Alteração dos dados contidos em transmissões ou arquivos.
		
	
	\end{enumerate}
	
	\section{Listas}
	
	\subsection{Lista 05 - Ataques Maliciosos, Ameaças e Vulnerabilidades}
	
	\begin{enumerate}
		\item O principal objetivo de um ciberataque é afetar um ou mais ativos de TI.  
		\begin{enumerate}[label=(\alph*)]
			\item \textcolor{blue}{Verdadeiro}
			\item Falso
		\end{enumerate}
		
		\item Qual dos seguintes descreve melhor a propriedade intelectual?  
		\begin{enumerate}[label=(\alph*)]
			\item Os itens que uma empresa protegeu por direitos autorais.
			\item Todas as patentes pertencentes a uma empresa.
			\item \textcolor{blue}{O conhecimento exclusivo que uma empresa possui.}
			\item O pessoal engajado em uma pesquisa exclusiva.
		\end{enumerate}
		
		\item Qual dos seguintes termos descreve melhor uma pessoa com muito pouca habilidade?  
		\begin{enumerate}[label=(\alph*)]
			\item Hacker
			\item \textcolor{blue}{Script kiddie}
			\item Cracker
			\item Aspirante
		\end{enumerate}
		
		\item Um(a) \textcolor{blue}{Spyware} é um software que captura tráfego enquanto ele atravessa uma rede.
		
		\item Qual tipo de ataque resulta em usuários legítimos sem acesso a um recurso de sistema?  
		\begin{enumerate}[label=(\alph*)]
			\item \textcolor{blue}{DoS}
			\item IPS
			\item Homem no meio
			\item Cavalo de Troia
		\end{enumerate}
		
		\item Um ataque de inundação de SYN inunda um alvo com pacotes de rede inválidos.  
		\begin{enumerate}[label=(\alph*)]
			\item \textcolor{blue}{Verdadeiro}
			\item Falso
		\end{enumerate}
		
		\item Qual das seguintes medidas pode proteger melhor seu computador contra worms?  
		\begin{enumerate}[label=(\alph*)]
			\item \textcolor{blue}{Instalar software antimalware}
			\item Configurar um firewall para bloquear todas as portas
			\item Criptografar todos os discos
			\item Impor senhas fortes para todos os usuários
		\end{enumerate}
		
		\item Um ataque de dicionário é um ataque simples, que conta principalmente com usuários que escolhem senhas fracas.  
		\begin{enumerate}[label=(\alph*)]
			\item \textcolor{blue}{Verdadeiro}
			\item Falso
		\end{enumerate}
		
		\item Qual tipo de ataque envolve capturar pacotes de dados de uma rede e retransmiti-los mais adiante para produzir um efeito não autorizado?  
		\begin{enumerate}[label=(\alph*)]
			\item Homem no meio
			\item Inundação de SYN
			\item \textcolor{blue}{Retransmissão (replay)}
			\item Smurf
		\end{enumerate}
		
		\item Um(a) \textcolor{blue}{Ameaça} é qualquer ação que possa danificar um ativo.
		
		\item Um(a) \textcolor{blue}{Vulnerabilidade} é qualquer falha que torne possível para uma ameaça causar danos a um computador ou uma rede.
		
		\item Qual tipo de malware é um programa autocontido que se replica e envia cópias de si mesmo para outros computadores, geralmente por uma rede?  
		\begin{enumerate}[label=(\alph*)]
			\item Vírus
			\item \textcolor{blue}{Verme}
			\item Cavalo de Troia
			\item Rootkit
		\end{enumerate}
		
	\end{enumerate}
	
	%-------------------------------------------------------------------------------%
	
	
	
	\subsection{Lista 06 - de Controle de Acesso}
	
	\begin{enumerate}
		\item Qual resposta descreve melhor o componente de autorização de controle de acesso?
		\begin{enumerate}[label=(\alph*)]
			\item Autorização é o método que um sujeito utiliza para solicitar acesso a um sistema.
			\item Autorização é o processo de criar e manter as políticas e os procedimentos necessários para garantir que informação apropriada esteja disponível quando uma organização for auditada.
			\item Autorização é a validação ou prova de que o sujeito que recebeu o acesso foi o mesmo que o solicitou.
			\item \textcolor{blue}{Autorização é o processo de determinar quem está aprovado para acesso para quais recursos.}
		\end{enumerate}
		
		\item Qual resposta descreve melhor o comportamento de identificação de controle de acesso?
		\begin{enumerate}[label=(\alph*)]
			\item Identificação é a validação ou prova de que o sujeito que recebeu o acesso foi o mesmo que o solicitou.
			\item \textcolor{blue}{Identificação é o método que um sujeito utiliza para solicitar acesso a um sistema.}
			\item Identificação é o processo de determinar quem está aprovado para acesso e para quais recursos.
			\item Identificação é o processo de criar e manter as políticas e os procedimentos necessários para garantir que informação apropriada esteja disponível quando uma organização for auditada.
		\end{enumerate}
		
		\item Qual resposta descreve melhor o componente de autenticação de controle de acesso?
		\begin{enumerate}[label=(\alph*)]
			\item \textcolor{blue}{Autenticação é a validação ou prova de que o sujeito que recebeu o acesso foi o mesmo que o solicitou.}
			\item Autenticação é o processo de criar e manter as políticas e os procedimentos necessários para garantir que informação apropriada esteja disponível quando uma organização for auditada.
			\item Autenticação é o processo de determinar quem está aprovado para acesso e para quais recursos.
			\item Autenticação é o método que um sujeito utiliza para solicitar acesso a um sistema.
		\end{enumerate}
		
		\item Qual resposta descreve melhor o componente de responsabilização de controle de acesso?
		\begin{enumerate}[label=(\alph*)]
			\item Responsabilização é a validação ou prova de que o sujeito que recebeu o acesso foi o mesmo que o solicitou.
			\item Responsabilização é o método que um sujeito utiliza para solicitar acesso a um sistema.
			\item \textcolor{blue}{Responsabilização é o processo de criar e manter as políticas e os procedimentos necessários para garantir que informação apropriada esteja disponível quando uma organização for auditada.}
			\item Responsabilização é o processo de determinar quem está aprovado para acesso e para quais recursos.
		\end{enumerate}
		
		\item Quando acessa uma rede, você recebe uma combinação de nome de usuário, senha, token, cartão inteligente ou biometria. Você, então, terá acesso autorizado ou negado pelo sistema. Este é um exemplo de .
		\begin{enumerate}[label=(\alph*)]
			\item Controles de acesso físico.
			\item \textcolor{blue}{Controles de acesso lógico.}
			\item Política de inclusão de grupo.
			\item Nenhuma das alternativas anteriores.
		\end{enumerate}
		
		\item Acesso físico, contorno de segurança e interceptação são exemplos de como os controles de acesso podem ser .
		\begin{enumerate}[label=(\alph*)]
			\item Roubados
			\item \textcolor{blue}{Comprometidos}
			\item Auditados
			\item Autorizados
		\end{enumerate}
		
		\item Desafios de controle de acesso incluem qual dos seguintes?
		\begin{enumerate}[label=(\alph*)]
			\item Perda de laptop
			\item Exploração de hardware
			\item Interceptação
			\item Exploração de aplicativos
			\item \textcolor{blue}{Todas as alternativas anteriores}
		\end{enumerate}
		
		\item Analise:
		\begin{itemize}
			\item I. Segurança física está associada à proteção de recursos através de controles como guardas, iluminação e detectores de movimento.
			\item II. Controle de acesso através de usuário e senha específicos em um determinado software aplicativo pode ser caracterizado como um controle físico.
			\item III. A segurança física está associada ao ambiente e a segurança lógica aos programas.
			\item IV. A segurança lógica deve ocorrer após a segurança física, através de softwares e protocolos.
		\end{itemize}
		São corretas as afirmações:
		\begin{enumerate}[label=(\alph*)]
			\item somente I, II e III
			\item somente I, II e IV
			\item somente II, III e IV
			\item \textcolor{blue}{somente I, III e IV}
			\item I, II, III e IV
		\end{enumerate}
		
		\item A respeito do controle de acesso a redes e aplicações, assinale, dentre as alternativas a seguir, a única que contém a ordem correta dos procedimentos lógicos atravessados por um usuário para acessar um recurso:
		\begin{enumerate}[label=(\alph*)]
			\item Autenticação, Identificação, Autorização e Auditoria.
			\item \textcolor{blue}{Identificação, Autenticação, Autorização e Auditoria.}
			\item Autorização, Identificação, Autenticação e Auditoria.
			\item Autorização, Autenticação, Identificação e Auditoria.
			\item Bloqueio, Autenticação, Autorização e Auditoria.
		\end{enumerate}
		
		\item A biometria se refere a várias técnicas de autenticação, para distinguir um indivíduo do outro, baseando-se nas características:
		\begin{enumerate}[label=(\alph*)]
			\item comportamentais, somente.
			\item físicas e/ou lógicas.
			\item \textcolor{blue}{físicas e/ou comportamentais.}
			\item físicas, somente.
			\item lógicas, somente.
		\end{enumerate}
		
		\item Obter confiança sobre a identidade de agentes ou integridade de dados em comunicação, baseando-se na posse de informação sigilosa (senha), dispositivos (smartcard), dado biométrico (impressão digital, retinal, etc) ou nas combinações destes elementos, trata-se do conceito de:
		\begin{enumerate}[label=(\alph*)]
			\item criptografia.
			\item \textcolor{blue}{autenticação.}
			\item assinatura digital.
			\item certificado digital.
			\item função de hash.
		\end{enumerate}
		
		\item Na ausência temporária do operador, o acesso ao computador por pessoa não autorizada pode ser evitado, de forma ideal, com a utilização de:
		\begin{enumerate}[label=(\alph*)]
			\item \textcolor{blue}{uma senha inserida na proteção de tela do Windows.}
			\item uma senha inserida no boot do computador.
			\item uma senha inserida para acesso ao disco rígido.
			\item desligamento do monitor, após alguns minutos de inatividade.
			\item desligamento do computador, sempre que o operador se retirar.
		\end{enumerate}
		
		\item Os métodos para implementação de um controle de acesso efetivo envolvem:
		\begin{enumerate}[label=(\alph*)]
			\item política de senhas, adoção de antivírus e firewall.
			\item \textcolor{blue}{identificação, autenticação, autorização e auditoria.}
			\item assinatura digital, detecção de intrusão e criptografia.
			\item política de senhas, plano de bloqueio e liberação.
			\item processo de login e rotinas de backup.
		\end{enumerate}
		
	\end{enumerate}
	
	
	
	
	
	
	
	%------------------------------------------------------------------------------%
	
	
	
	\subsection{Lista 07 - Gerenciamento de Riscos e Plano de Continuidade de Negócios}
	
	\begin{enumerate}
		\item De acordo com o PMI, qual termo descreve a lista de riscos identificados?
		\begin{enumerate}[label=(\alph*)]
			\item Lista de verificação de riscos
			\item \textcolor{blue}{Registrador de riscos}
			\item Metodologia de riscos
			\item Lista de atenuação
		\end{enumerate}
		
		\item Que tipo de análise de risco usa fórmulas e valores numéricos para indicar seriedade de risco?
		\begin{enumerate}[label=(\alph*)]
			\item Análise objetiva de risco
			\item Análise qualitativa de risco
			\item Análise subjetiva de risco
			\item \textcolor{blue}{Análise quantitativa de risco}
		\end{enumerate}
		
		\item Qual tipo de análise de risco usa classificação relativa?
		\begin{enumerate}[label=(\alph*)]
			\item Análise objetiva de risco
			\item \textcolor{blue}{Análise qualitativa de risco}
			\item Análise subjetiva de risco
			\item Análise quantitativa de risco
		\end{enumerate}
		
		\item Qual valor de análise de risco representa a probabilidade anual de uma perda?
		\begin{enumerate}[label=(\alph*)]
			\item EF
			\item SLE
			\item ALE
			\item \textcolor{blue}{ARO}
		\end{enumerate}
		
		\item Qual opção de resposta a risco descreveria melhor a realização de um seguro contra incêndio?
		\begin{enumerate}[label=(\alph*)]
			\item Aceitar
			\item Atenuar
			\item \textcolor{blue}{Transferir}
			\item Evitar
		\end{enumerate}
		
		\item Qual resposta a risco seria mais apropriada se a possibilidade do impacto de um risco se tornar realidade for desprezível?
		\begin{enumerate}[label=(\alph*)]
			\item \textcolor{blue}{Aceitar}
			\item Atenuar
			\item Transferir
			\item Evitar
		\end{enumerate}
		
		\item Qual das seguintes afirmações descreve melhor a relação entre um BCP e um DRP?
		\begin{enumerate}[label=(\alph*)]
			\item Um BCP é obrigatório, mas um DRP não.
			\item \textcolor{blue}{Um DRP é um componente de um BCP.}
			\item Um DRP é obrigatório, mas um BCP não.
			\item Um BCP é um componente de um DRP.
		\end{enumerate}
		
		\item Qual termo é usado para indicar a quantidade de perda de dados aceitável?
		\begin{enumerate}[label=(\alph*)]
			\item RAI
			\item ROI
			\item RTO
			\item \textcolor{blue}{RPO}
		\end{enumerate}
		
		\item Qual metodologia de avaliação de risco é comercializada como abordagem autodirecionada e tem duas edições diferentes para organizações de tamanhos diferentes?
		\begin{enumerate}[label=(\alph*)]
			\item CRAMM
			\item \textcolor{blue}{OCTAVE}
			\item NIST
			\item EBIOS
		\end{enumerate}
		
		\item Um Analista de Segurança de Informações do Tribunal de Justiça está redigindo um documento que estabelece ações de monitoração de riscos e prevenção de problemas, de forma a evitar interrupções em operações do negócio. Esse documento será parte integrante
		\begin{enumerate}[label=(\alph*)]
			\item do Plano de Recuperação de Desastres.
			\item \textcolor{blue}{do Plano de Continuidade dos Negócios.}
			\item do Plano de Segurança da Informação.
			\item da Estratégia de Serviços de TI.
		\end{enumerate}
		
		\item No que se refere ao plano de continuidade de negócios, assinale a opção correta.
		\begin{enumerate}[label=(\alph*)]
			\item Os objetivos do plano em tela incluem evitar a interrupção das atividades do negócio, proteger os processos críticos contra o acesso de pessoas estranhas ao ambiente e assegurar a retomada dos processos em tempo hábil, caso necessário.
			\item A existência de um gestor específico para cada plano de continuidade é desvantajoso, visto que causa aumentos significativos dos custos dos planos como um todo.
			\item Os planos de continuidade do negócio devem ser testados e atualizados infrequentemente, já que a realização regular dessas ações acarreta o aumento significativo dos custos dos planos.
			\item \textcolor{blue}{A estrutura de planejamento para continuidade de negócios deve abranger os ativos e os recursos críticos para uma eventual utilização dos procedimentos de emergência, recuperação e ativação.}
		\end{enumerate}
		
		\item O plano de continuidade do negócio deve
		\begin{enumerate}[label=(\alph*)]
			\item ter a mesma definição e desenvolvimento para todas as organizações e utilizar uma abordagem genérica, já que dessa forma poderá abranger todos os aspectos críticos que causam impactos negativos ao negócio.
			\item \textcolor{blue}{ser eficiente e eficaz, ser mantido atualizado e ser testado periodicamente contando com a participação de todos os envolvidos.}
			\item ser do conhecimento apenas da alta administração que deve conhecer e aprovar as ameaças e riscos que estão fora do escopo do plano.
			\item ser elaborado de forma que possibilite seu funcionamento em condições perfeitas, em nível otimizado, garantindo que não haja a possibilidade de incidentes que gerem impactos financeiros ou operacionais.
		\end{enumerate}
		
		\item O Plano de Continuidade do Negócio:
		\begin{enumerate}[label=(\alph*)]
			\item não precisa ser testado antes que se torne realmente necessário, pois testes por si só implicam em riscos aos ativos de informação.
			\item prioriza e estabelece as ações de implantação como resultado de uma ampla análise de risco.
			\item \textcolor{blue}{define uma ação de continuidade imediata e temporária.}
			\item precisa ser contínuo, evoluir com a organização, mas não precisa ser gerido sob a responsabilidade de alguém como os processos organizacionais.
		\end{enumerate}
		
		\item Considerando a TI, as empresas devem ter constante preocupação com os riscos, que se concretizados, podem vir a prejudicar suas atividades. Dessa forma, a gestão de riscos é uma atividade de grande importância na condução dos negócios de uma empresa. Na maioria dos casos, a primeira etapa a ser realizada na gestão de riscos é a identificação dos riscos, que consiste em
		\begin{enumerate}[label=(\alph*)]
			\item elaborar os planos de contingência, cujo objetivo é obter um controle preciso dos riscos presentes.
			\item minimizar os problemas que possam surgir, eventualmente, em função dos riscos existentes.
			\item \textcolor{blue}{detectar os perigos potenciais que possam vir a prejudicar as operações da empresa, como a execução de um projeto de TI.}
			\item registrar todas as ações tomadas no decorrer da concretização de um risco de forma a evitar problemas semelhantes no futuro.
		\end{enumerate}
		
	\end{enumerate}
	
	
	
	
	
	
	
	
	%-----------------------------------------------------------------------------%
	
	
	
	
	\subsection{Lista 08 - Auditoria de Sistemas}
	
	\begin{enumerate}
		\item Qual dos seguintes é um exemplo de um nível de permissividade?
		\begin{enumerate}[label=(\alph*)]
			\item Prudente
			\item Permissivo
			\item Paranóico
			\item Promíscuo
			\item \textcolor{blue}{Todas as alternativas anteriores}
		\end{enumerate}
		
		\item Uma auditoria examina se os controles de segurança são apropriados, estão instalados corretamente e são/estão .
		\begin{enumerate}[label=(\alph*)]
			\item Atualizados
			\item \textcolor{blue}{Cuidando de seu objetivo}
			\item Autorizados
			\item Econômicos
		\end{enumerate}
		
		\item Uma é um padrão usado para medir quão efetivo seu sistema é em relação a expectativas do setor.
		\begin{enumerate}[label=(\alph*)]
			\item Objetivo de controle
			\item Configuração
			\item \textcolor{blue}{Padrão de referência (benchmark)}
			\item Política
		\end{enumerate}
		
		\item Atividades de pós-auditoria incluem qual das seguintes?
		\begin{enumerate}[label=(\alph*)]
			\item Apresentar descobertas à gerência
			\item Analisar dados
			\item Entrevistas de saída
			\item Análise de descobertas do auditor
			\item \textcolor{blue}{Todas as alternativas anteriores}
		\end{enumerate}
		
		\item é usado quando não é tão crítico detectar e responder a incidentes imediatamente.
		\begin{enumerate}[label=(\alph*)]
			\item \textcolor{blue}{Monitoramento que não seja em tempo real}
			\item Um controle de acesso lógico
			\item Monitoramento em tempo real
			\item Nenhuma das alternativas anteriores
		\end{enumerate}
		
		\item Uma plataforma comum para capturar e analisar entradas de histórico é .
		\begin{enumerate}[label=(\alph*)]
			\item Sistema de detecção de intrusos (IDS)
			\item Honeypot
			\item \textcolor{blue}{Informação de Segurança e Gerenciamento de Evento (SIEM - Security Information and Event Management)}
			\item HIPAA
		\end{enumerate}
		
		\item Em métodos ., o IDS compara tráfego atual com padrões de atividade consistente com aqueles de uma intrusão de rede conhecida via casamento de padrão e casamento de estado.
		\begin{enumerate}[label=(\alph*)]
			\item \textcolor{blue}{Baseados em assinatura}
			\item Baseados em anomalia
			\item De varredura heurística
			\item Todas as alternativas anteriores
		\end{enumerate}
		
		\item Isolamento de computador é o isolamento de redes internas e o estabelecimento de um(a) .
		\begin{enumerate}[label=(\alph*)]
			\item HIDS
			\item \textcolor{blue}{DMZ}
			\item IDS
			\item IPS
		\end{enumerate}
		
		\item A análise do sistema para descobrir o máximo possível sobre a organização, seus sistemas e redes é conhecida como .
		\begin{enumerate}[label=(\alph*)]
			\item Teste de penetração
			\item Teste de vulnerabilidade
			\item Mapeamento de rede
			\item \textcolor{blue}{Reconhecimento}
		\end{enumerate}
		
	\end{enumerate}
	
	
	
	
\end{document}
