\documentclass[12pt]{article}
\usepackage{amsmath}
\usepackage{cite}
\usepackage[utf8]{inputenc}
\usepackage[portuguese]{babel}
\usepackage{hyperref}
\hypersetup{colorlinks=false, pdfborder={0 0 0}}
\usepackage{fancyhdr}
\usepackage{graphicx}


\title{Segurança da Informação: Computação Quântica e Criptografia Pós-Quântica}
\author{João Pedro Rodrigues Leite}

\begin{document}
	
	\maketitle
	
	\section{Introdução}
	Nos últimos anos, o campo da \textit{Segurança da Informação} tem enfrentado grandes desafios devido ao avanço da tecnologia quântica. Empresas como IBM, Microsoft e Google estão envolvidas e empenhadas no desenvolvimento de computação quântica. O desenvolvimento de computadores quânticos, que utilizam princípios da física/mecânica quântica como a superposição e o emaranhamento, princípios nos quais não podem ser explorados na computação convencional. o princípio de superposição pode ser entendido por exemplo como a representação de um bit quântico ou qubit em vários estados, podendo então assumir valores como 0, 1 ou ambos ao mesmo tempo, já o emaranhamento cria uma interdepedência entre as partículas ou os qubits independentemente da distância entre eles, então ao fazer a medição do valor de um qubit imediatamente já será possível identificar o valor assumido pelo outro qubit que está interligado, com isso, os computadores quânticos podem realizar operações em paralelo, e ter o potencial de quebrar a criptografia convencional usada para proteger dados sensíveis nas comunicações digitais \cite{mitra2017quantum}. Esses sistemas representam uma ameaça significativa, pois, enquanto a criptografia convencional depende da complexidade computacional de realizar cálculos matemáticos, os computadores quânticos podem realizar operações paralelas em escala exponencial, comprometendo a segurança de muitos algoritmos de criptografia que são utilizados hoje.
	
	\section{Ameaças da Computação Quântica}
	Uma das principais ameaças que surgem com o desenvolvimento dos computadores quânticos é a estratégia conhecida como \textit{store now, decrypt later} (armazenar agora, decifrar depois) \cite{argetsinger2024promise}. Essa técnica consiste no armazenamento de dados criptografados com a esperança de que, no futuro, quando os computadores quânticos forem capazes de quebrar algoritmos criptográficos atuais utilizando algoritmos como o de Shor, esses dados possam ser decifrados. Esse cenário coloca em risco a integridade e confidencialidade das comunicações digitais.
	
	\section{Soluções Criptográficas Pós-Quânticas}
	Diante desse contexto, novas técnicas de \textit{criptografia pós-quântica} (CPQ) estão sendo desenvolvidas para mitigar os riscos impostos pela computação quântica. A CPQ visa criar algoritmos que sejam resistentes a ataques baseados em computadores quânticos, utilizando métodos que permanecem permitem os dados sejam mantidos em segurança mesmo com a capacidade de processamento exponencial computadores quânticos \cite{nandhini2022extensive}.
	
	\section{Comparação entre Criptografia Clássica e Pós-Quântica}
	A diferença entre a criptografia clássica e a pós-quântica reside nos fundamentos matemáticos que protegem os dados. A criptografia clássica, como o RSA, que é amplamente utilizado na geração de chaves assimétricas como os certificados digitais, nas comunicações SSH para autenticação ou até mesmo na camada SSL/TLS para manter a comunicação segura, sua segurança baseia-se na complexidade da fatoração de números grandes e no logaritmo discreto, problema que é vulnerável a algoritmos quânticos como o algoritmo desenvolvido por Peter Shor em 1994 \cite{shor1994algorithms}. Há também outro algoritmo que foi desenvolvido por Lov Grover em 1996, e que também pode causar um impacto na criptografia convencial, especificamente nas que são baseadas em chaves simétricas como o AES, 3DES entre outros. Estima-se que esse algoritmo nas chaves simétricas reduziria a segurança delas pela metade, e uma das soluções encontradas para reduzir os impactos seria duplicar o tamanho da chave para manter a segurança dos dados\cite{grover1996fast}. Em contrapartida, os algoritmos de criptografia pós-quântica são projetados para serem resistentes a esses tipos de ataques, garantindo a segurança das informações em um cenário de computação quântica \cite{Callan2024}.
	
	\section{Conclusão}
	Com o iminente avanço da computação quântica, a segurança da informação enfrenta um de seus maiores desafios. A adoção de algoritmos de criptografia pós-quântica é essencial para garantir a proteção dos dados nas comunicações digitais.
	
	\bibliographystyle{plain}
	\bibliography{referencias}
\end{document}
