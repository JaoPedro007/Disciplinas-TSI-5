% PROPOSTA DE TRABALHO DE CONCLUSÃO DE CURSO-----------------------------------------------------------

\chapter{PROPOSTA DE TRABALHO DE CONCLUSÃO DE CURSO}
\label{chap:proposta}

\section{TÍTULO}
\label{sec:titulo}
% Informe o título do trabalho-------------------------------------------------------------------------

O título desta proposta de conclusão de curso (TCC) é ``Segurança em comunicações pós-quânticas''.
% título é um texto, com poucas palavras, que deve expressar claramente: O objeto de investigação relativo ao tema e o que vai fazer (substitua este texto pelo título do trabalho).
%------------------------------------------------------------------------------------------------------

\section{MODALIDADE DO TRABALHO}
\label{sec:modalidade}
% Indique a Modalidade do Trabalho---------------------------------------------------------------------
% Opções:
% - Pesquisa
% - Desenvolvimento de Sistemas

A presente proposta de TCC enquadra-se na categoria de trabalho científico aplicado.

%Trabalho Tecnológico ou Trabalho Científico Aplicado
%------------------------------------------------------------------------------------------------------

\section{ÁREA DO TRABALHO}
\label{sec:area}
% Indique a Área do Trabalho---------------------------------------------------------------------------
%Definir a área em que o trabalho está incluído (substitua este texto pela área do trabalho).

Esta proposta de TCC está inserida na área de segurança da informação, especificamente na criptografia pós-quântica. Esta é uma subárea emergente que aborda a criação e implementação de métodos de criptografia que possam resistir aos ataques de computadores quânticos, os quais têm potencial para quebrar os esquemas criptográficos convencionais atualmente em uso. À medida que a computação quântica evolui e se aproxima de uma aplicação prática, o desenvolvimento de estratégias de criptografia que possam garantir a segurança das comunicações digitais contra tais ameaças torna-se crucial. 

%------------------------------------------------------------------------------------------------------
{\color{cyan}
\section{RESUMO}
\label{sec:resumo}
% Resumo do Trabalho-----------------------------------------------------------------------------------
% (máximo de 200 palavras)
%Um resumo deve informar a essência do projeto de maneira resumida, mas completa. Os leitores devem ter uma ideia razoavelmente clara do projeto após ter lido o resumo. Basicamente deve-se colocar informações referentes a finalidade da pesquisa, procedimentos que serão utilizados, observações e dados a serem coletados, resultados esperados (substitua este texto pelo resumo do trabalho).
%------------------------------------------------------------------------------------------------------
Esta proposta de trabalho de conclusão de curso(TCC) aborda a crescente preocupação na área da segurança da informação em relação à possível ameaça representada pelos computadores quânticos à criptografia convencional. Com o avanço da pesquisa e desenvolvimento de tecnologia quântica, torna-se evidente que algoritmos de criptografia atualmente considerados seguros podem ser quebrados de forma eficiente por esses dispositivos revolucionários. Diante desse cenário, surge a necessidade de identificar e apresentar novas técnicas criptográficas resistentes aos ataques quânticos. Há a necessidade de garantir a segurança das comunicações digitais em um futuro onde os computadores quânticos podem comprometer a eficácia dos algoritmos convencionais. Além disso, é discutido o desafio do "Store Now, Decrypt Later" (SNDL), no qual os dados são armazenados aguardando um momento futuro para serem decifrados. Pesquisas sobre trabalhos relacionados serão realizados de modo a identificar e apresentar os algoritmos relevantes e seus fundamentos. Espera-se que com esse trabalho haja uma conscientização da ameaça e dos problemas enfrentados na segurança da informação com a iminente chegada dos computadores quânticos.
}
