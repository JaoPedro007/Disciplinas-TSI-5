\documentclass[aspectratio=169]{beamer}
\usetheme{Warsaw}
\usepackage[utf8]{inputenc}
\usepackage{graphicx}
\usepackage{styles/style}

\title{Segurança em comunicações Pós-Quânticas}
\author{João Pedro Rodrigues Leite}
\date[Toledo, 2024]

\institute{
	Orientador: Prof. Dr. Fábio Engel de Camargo
	\\Universidade Tecnológica Federal do Paraná ({\bf{UTFPR}})
	\\Curso de Sistemas para Internet
}

\titlegraphic{%
	\begin{minipage}{0.5\textwidth}
		\includegraphics[width=\linewidth]{figs/logo-utfpr.pdf}
	\end{minipage}
}

\begin{document}
	
	\begin{frame}
		\titlepage
	\end{frame}
	
	\begin{frame}{Sumário}
		\tableofcontents
	\end{frame}
	
	\section{Contextualização}
	\begin{frame}{Contextualização}
		\begin{itemize}
			\item Avanços significativos na computação quântica ameaçam a segurança da criptografia convencional.
			\item Computadores quânticos podem realizar operações em paralelo, explorando superposição e emaranhamento.
			\item Essa capacidade pode comprometer a confidencialidade e integridade das comunicações digitais.
			\item A criptografia pós-quântica (CPQ) emerge como uma resposta a essas ameaças.
		\end{itemize}
	\end{frame}
	
	\begin{frame}{História da Computação Quântica}
		\begin{itemize}
			\item 1981: Richard Feynman propõe a ideia de um computador baseado em princípios quânticos.
			\item 1985: David Deutsch formaliza o conceito de um computador quântico universal.
			\item 1994: Peter Shor desenvolve um algoritmo quântico que ameaça a criptografia baseada em fatoração.
			\item 1996: Lov Grover apresenta um algoritmo quântico eficiente para busca em bases de dados.
		\end{itemize}
	\end{frame}
	
	\section{Definição do Problema}
	\begin{frame}{Definição do Problema}
		\begin{itemize}
			\item A estratégia ``store now, decrypt later'' é uma ameaça emergente.
			\item Dados criptografados hoje podem ser vulneráveis a futuros avanços na computação quântica.
			\item A necessidade de algoritmos que resistam a ataques quânticos é urgente.
		\end{itemize}
	\end{frame}
	
	\section{Objetivos}
	\begin{frame}{Objetivo Geral}
		\begin{itemize}
			\item Estudar e apresentar os problemas e soluções de criptografia pós-quântica existentes de uma maneira mais acessível e compreensível, a fim de garantir a segurança das comunicações digitais.
		\end{itemize}
	\end{frame}
	
	\begin{frame}{Objetivos Específicos}
		\begin{itemize}
			\item Identificar e catalogar os principais algoritmos de criptografia       pós-quântica.
			\item Avaliar as bases teóricas e contextos de aplicação desses             algoritmos.
                \item Apresentar uma comparação sobre os algoritmos de criptografia pós-quântica entre si em termos de complexidade computacional e praticabilidade.
			\item Elaborar recomendações para desenvolvedores web sobre o uso de CPQ.
		\end{itemize}
	\end{frame}
	
	\section{Metodologia}
	\begin{frame}{Metodologia}
		\begin{itemize}
			\item Revisão bibliográfica sobre algoritmos de criptografia pós-quântica.
			\item Comparar os algoritmos em termos de desempenho e segurança.
			\item Desenvolvimento de recomendações para a adoção de CPQ em sistemas web.
		\end{itemize}
	\end{frame}
	
	\section{Resultados Esperados}
	\begin{frame}{Resultados Esperados}
		\begin{itemize}
			\item Identificação dos algoritmos de CPQ mais promissores.
			\item Revisão das vantagens e limitações desses algoritmos.
			\item Recomendações práticas para a implementação de CPQ em sistemas digitais.
		\end{itemize}
	\end{frame}

        \section{Cronograma}
        \begin{frame}{Cronograma}
            \begin{itemize}
                \item \textbf{Outubro (2024):} Revisão dos apontamentos da banca.
                \item \textbf{Novembro (2024):} Revisão bibliográfica e redação do projeto de TCC.
                \item \textbf{Dezembro (2024):} Defesa do projeto de TCC e início da escrita da monografia.
                \item \textbf{Janeiro (2025):} Continuação da escrita da monografia e elaboração da apresentação final.
                \item \textbf{Fevereiro (2025):} Finalização da monografia, preparação da apresentação, revisão geral do trabalho, e defesa final do TCC.
            \end{itemize}
        \end{frame}
        
	\section{Conclusão}
	\begin{frame}{Conclusão}
		\begin{itemize}
			\item A computação quântica representa uma ameaça real à segurança digital.
			\item A adoção de criptografia pós-quântica é essencial para garantir a proteção de dados no futuro.
			\item Este trabalho contribuirá para a compreensão e adoção de CPQ.
		\end{itemize}
	\end{frame}




	
\end{document}
