\documentclass{article}
\usepackage{amsmath}
\usepackage{booktabs}
\usepackage{graphicx}
\usepackage{listings}
\usepackage{url}

\begin{document}
	
	\begin{enumerate}
		\item \textbf{Questão 1:}
		
		A inequação é representada por:
		\begin{equation}
			\frac{h_{ij} \cdot P_{T_i} \cdot d(i, j) - \alpha}{N_0 + \sum_{\substack{k=1 \\ k \neq i}}^{\tau} h_{kj} \cdot P_{T_i} \cdot d(k, j) - \alpha} \geq \gamma
		\end{equation}
		
		\item \textbf{Questão 2:}
		
		\begin{table}[htbp]
			\centering
			\caption{Relação de dispositivos que podem transmitir em simultâneo a $i$.}
			\begin{tabular}{cccc}
				\toprule
				Transmissão & $i$ & $j$ & $\beta_i$ \\
				\midrule
				(0,6) & 0 & 6 & \{3, 5\} \\ \hline
				(1,6) & 1 & 6 & \{2, 4, 5, 8\} \\ \hline
				(2,8) & 2 & 8 & \{1, 3, 5\} \\ \hline
				(3,5) & 3 & 5 & \{0, 2, 6, 8\} \\ \hline
				(4,7) & 4 & 7 & \{1, 6\} \\ \hline
				(5,3) & 5 & 3 & \{0, 1, 2, 6, 8\} \\
				\bottomrule
			\end{tabular}
		\end{table}
		
		\item \textbf{Questão 3:}
		
		Para este exercício, foram selecionados os seguintes artigos e site:
		\begin{itemize}
			\item No artigo \cite{de2020esquemas}, estuda-se dois algoritmos, mais precisamente, os esquemas de assinatura digital qTESLA e Crystals-Dilithium, tendo como ferramenta principal, no que concerne à implementação não otimizada dos mesmos, o software SageMath.
			\item No artigo \cite{amorim2020comparaccao}, o surgimento do primeiro computador quântico funcional levanta preocupações sobre a quebra da segurança de algoritmos criptográficos tradicionais. Este trabalho avalia propostas para esse novo padrão, destacando suas performances na segunda rodada do Processo de Padronização Criptográfica Pós-Quântica do NIST.
			
			\item No artigo \cite{misoczki2008criptografia}, analisa e implementa o sistema criptográfico McEliece, o qual é classificado como pós-quântico por se tratar de um sistema seguro perante as abordagens atuais de ataques utilizadno computação quântica.
			
			\item No site \cite{prodesp}, a criptografia pós-quântica envolve algoritmos e fundamentos matemáticos distintos.
			
		\end{itemize}
		

			\bibliographystyle{plainnat}
			\bibliography{references} 


		
		
		\item \textbf{Questão 4:}
		\begin{itemize}
			\item Nível 1
			\begin{itemize}
				\item Nível 2
				\begin{itemize}
					\item Nível 3
				\end{itemize}
			\end{itemize}
		\end{itemize}
		
		\item \textbf{Questão 5:}
		\lstset{language=Java}
		\begin{lstlisting}
			public class HelloWorld {
				public static void main(String[] args) {
					System.out.println("Hello, world!");
				}
			}
		\end{lstlisting}
		
		\item \textbf{Questão 6:}
		\begin{figure}[htbp]
			\centering
			\begin{minipage}[b]{0.40\textwidth}
				\centering
				\includegraphics[width=\textwidth,height=2.5cm]{./utfpr_logo.png}
				{(a) Logo da UTFPR.}
			\end{minipage}\hfill
			\begin{minipage}[b]{0.40\textwidth}
				\centering
				\includegraphics[width=\textwidth,height=2.5cm]{./tsi_logo.png}
				{(b) Logo de TSI.}
			\end{minipage}
			\caption{Exemplos de logos.}
		\end{figure}
		
	\end{enumerate}
	
	
\end{document}
